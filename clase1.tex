\documentclass{memoir}
\usepackage{enumerate}
\begin{document}
Profesor Luis Carrera

Linux servicios (inicio y reinicio)
 
 Windows como cliente y servicio.

 useradd lcarrera
 password lcarrera


 for i in 1 2 3 4 
 do
% useradd user$i
 echo 

chpassword

Linux

Sistema operativo que replica las características del sistema operativo UNIX

Sistema operativo

Programa que se encarga de administrar los recursos de un sistema de cómputo, tales como: CPU, RAM, discos duros, etc.
\section{Características}

\begin{enumerate}
\item Multitarea

Capacidad de realizar múltiples procesos simultáneamente.

\item Multiusuario

Capacidad de ejecutar los procesos de múltiples usuarios simultáneamente.

\item Multiplatafora	

Capacidad de ejecutar el sistema operativo en diferentes plataformas de hardware (Intel, Risc, Sparc).
\end{enumerate}

Linux consiste, solamente en el núcleo del sistema operativo, si a este núcleo le añadimos un conjunto de aplicaciones, creamos una distribución tal como RedHat, Fedora, CentOS, Debian, Ubuntu, Slackware, Mandriva, etc.

Sistema de cómputo

Sistema que permite realizar tareas de procesamiento de datos, consisten básicamente en un CPU (Unidad Central de Procesos), RAM (Memoria de acceso aleatorio) y disco duro (sistema de archivos).

El acceso a un sistema de cómputo, requiere que se tenga asignado un usuario y una contraseña de acceso.

El acceso puede ser:

Local

El usuario se encuentra frente al teclado y el monitor conectado directamente al sistema.

Remoto

El usuario accede a través de un terminal físico o emulador de terminal.

Los terminales físicos se conectan al puerto de comunicaciones del sistema de cómputo, su extensión (cuán separado está el terminal del sistema de cómputo) es limitado generalmente una decena de metros y obliga el uso de repetidores.

Los emuladores del terminal son programas que se ejecutan en cualquier sistema operativo, el único requisito es que el sistema tenga configurado el lenguaje de comunicación de red TCP/IP, el emulador se conectará al sistema de cómputo usando los protocolos TELNET (en desuso) o SSH para iniciar una sesión remota, a partir de aquí el comportamiento es igual como si estuviera conectado directamente al sistema de cómputo.

TCP/IP

Servicio Puerta de acceso

http 80
https 443
ssh 22
telnet 23

cat /etc/services

Acceso al servidor

IP: 172.17.2.229
usuario: user2
clave: user2

id

uid=503(user2) gid=503(user2) grupos=503(user2) context=unconfined_u:unconfined_r:unconfined_t:s0-s0:c0.c1023

hostname

luiscarrera.pe

uname (unixname)

Linux

uname -r

2.6.32-358.el6.x86_64 (Versión del Kernel de Linux Mint)

date

sáb ene 27 16:03:04 PET 2018

who

root     tty1         2018-01-27 15:06 (:0)
root     pts/0        2018-01-27 15:07 (:0.0)
root     pts/1        2018-01-27 15:53 (:0.0)
user4    pts/2        2018-01-27 15:56 (172.17.3.252)
user9    pts/3        2018-01-27 16:01 (172.17.2.232)
user7    pts/4        2018-01-27 15:56 (172.17.2.231)
user6    pts/5        2018-01-27 15:56 (172.17.0.225)
user3    pts/6        2018-01-27 15:56 (172.17.3.220)
user2    pts/7        2018-01-27 15:56 (172.17.2.219)
user5    pts/8        2018-01-27 15:57 (172.17.3.223)
user10   pts/9        2018-01-27 15:58 (172.17.2.233)
user8    pts/10       2018-01-27 15:58 (172.17.0.2)
user1    pts/11       2018-01-27 16:04 (172.17.2.217)

w

USER     TTY      FROM              LOGIN@   IDLE(tiempo inactivo)   JCPU   PCPU WHAT(programa que está ejecutano)
root     tty1     :0               15:06    1:27m 24.84s 24.84s /usr/bin/Xorg :
root     pts/0    :0.0             15:07   12:38  18.70s 18.55s /usr/bin/python
root     pts/1    :0.0             15:53    9.00s  0.05s  0.02s -bash
user4    pts/2    172.17.3.252     15:56    1:15   0.09s  0.09s -bash
user9    pts/3    172.17.2.232     16:01    3:43   0.02s  0.02s -bash
user7    pts/4    172.17.2.231     15:56    2:08   0.02s  0.02s -bash
user6    pts/5    172.17.0.225     15:56    6.00s  0.04s  0.04s -bash
user3    pts/6    172.17.3.220     15:56    2:09   0.05s  0.05s -bash
user2    pts/7    172.17.2.219     15:56    0.00s  0.38s  0.00s w
user5    pts/8    172.17.3.223     15:57    2.00s  0.04s  0.04s -bash
user10   pts/9    172.17.2.233     15:58    2.00s  0.03s  0.03s -bash
user8    pts/10   172.17.0.2       15:58    3.00s  0.04s  0.04s -bash
user1    pts/11   172.17.2.217     16:04   28.00s  0.02s  0.02s -bash


Shell

Permite al usuario interactuar con la máquina.

who -r

 `run-level' 5 2018-01-27 14:38

 man who

cat /etc/resolv.conf

# Generated by NetworkManager
search pe


# No nameservers found; try putting DNS servers into your
# ifcfg files in /etc/sysconfig/network-scripts like so:
#
# DNS1=xxx.xxx.xxx.xxx
# DNS2=xxx.xxx.xxx.xxx
# DOMAIN=lab.foo.com bar.foo.com
nameserver 172.17.3.254
nameserver 8.8.8.8
nameserver 8.8.4.4

cat /etc/passwd

root:x:0:0:root:/root:/bin/bash
bin:x:1:1:bin:/bin:/sbin/nologin
daemon:x:2:2:daemon:/sbin:/sbin/nologin
adm:x:3:4:adm:/var/adm:/sbin/nologin
lp:x:4:7:lp:/var/spool/lpd:/sbin/nologin
sync:x:5:0:sync:/sbin:/bin/sync
shutdown:x:6:0:shutdown:/sbin:/sbin/shutdown
halt:x:7:0:halt:/sbin:/sbin/halt
mail:x:8:12:mail:/var/spool/mail:/sbin/nologin
uucp:x:10:14:uucp:/var/spool/uucp:/sbin/nologin
operator:x:11:0:operator:/root:/sbin/nologin
games:x:12:100:games:/usr/games:/sbin/nologin
gopher:x:13:30:gopher:/var/gopher:/sbin/nologin
ftp:x:14:50:FTP User:/var/ftp:/sbin/nologin
nobody:x:99:99:Nobody:/:/sbin/nologin
dbus:x:81:81:System message bus:/:/sbin/nologin
rpc:x:32:32:Rpcbind Daemon:/var/cache/rpcbind:/sbin/nologin
usbmuxd:x:113:113:usbmuxd user:/:/sbin/nologin
abrt:x:173:173::/etc/abrt:/sbin/nologin
pegasus:x:66:65:tog-pegasus OpenPegasus WBEM/CIM services:/var/lib/Pegasus:/sbin/nologin
cimsrvr:x:134:134:tog-pegasus OpenPegasus WBEM/CIM services:/var/lib/Pegasus:/sbin/nologin
hsqldb:x:96:96::/var/lib/hsqldb:/sbin/nologin
oprofile:x:16:16:Special user account to be used by OProfile:/home/oprofile:/sbin/nologin
vcsa:x:69:69:virtual console memory owner:/dev:/sbin/nologin
rtkit:x:499:497:RealtimeKit:/proc:/sbin/nologin
avahi-autoipd:x:170:170:Avahi IPv4LL Stack:/var/lib/avahi-autoipd:/sbin/nologin
apache:x:48:48:Apache:/var/www:/sbin/nologin
saslauth:x:498:76:"Saslauthd user":/var/empty/saslauth:/sbin/nologin
postfix:x:89:89::/var/spool/postfix:/sbin/nologin
rpcuser:x:29:29:RPC Service User:/var/lib/nfs:/sbin/nologin
nfsnobody:x:65534:65534:Anonymous NFS User:/var/lib/nfs:/sbin/nologin
haldaemon:x:68:68:HAL daemon:/:/sbin/nologin
gdm:x:42:42::/var/lib/gdm:/sbin/nologin
tomcat:x:91:91:Apache Tomcat:/usr/share/tomcat6:/sbin/nologin
mysql:x:27:27:MySQL Server:/var/lib/mysql:/bin/bash
ntp:x:38:38::/etc/ntp:/sbin/nologin
amandabackup:x:33:6:Amanda user:/var/lib/amanda:/bin/bash
memcached:x:497:495:Memcached daemon:/var/run/memcached:/sbin/nologin
pulse:x:496:494:PulseAudio System Daemon:/var/run/pulse:/sbin/nologin
webalizer:x:67:67:Webalizer:/var/www/usage:/sbin/nologin
sshd:x:74:74:Privilege-separated SSH:/var/empty/sshd:/sbin/nologin
postgres:x:26:26:PostgreSQL Server:/var/lib/pgsql:/bin/bash
dovecot:x:97:97:Dovecot IMAP server:/usr/libexec/dovecot:/sbin/nologin
dovenull:x:495:491:Dovecot's unauthorized user:/usr/libexec/dovecot:/sbin/nologin
tcpdump:x:72:72::/:/sbin/nologin
Alumno:x:500:500::/home/Alumno:/bin/bash
lcarrera:x:501:501::/home/lcarrera:/bin/bash
user1:x:502:502::/home/user1:/bin/bash
user2:x:503:503::/home/user2:/bin/bash
user3:x:504:504::/home/user3:/bin/bash
user4:x:505:505::/home/user4:/bin/bash
user5:x:506:506::/home/user5:/bin/bash
user6:x:507:507::/home/user6:/bin/bash
user7:x:508:508::/home/user7:/bin/bash
user8:x:509:509::/home/user8:/bin/bash
user9:x:510:510::/home/user9:/bin/bash
user10:x:511:511::/home/user10:/bin/bash

head -n /etc/passwd

veo las primeras n líneas del archivo. Sin su bandera muestra 10

tail -n /etc/passwd

veo las últimas n líneas del archivo

wc /etc/passwd
58   99 2925 /etc/passwd

wc -l /etc/passwd

58 /etc/passwd

wc -w /etc/passwd

99 /etc/passwd

wc -c /etc/passwd

2925 /etc/passwd

grep

gret regular expression

grep user /etc/passwd

usbmuxd:x:113:113:usbmuxd user:/:/sbin/nologin
oprofile:x:16:16:Special user account to be used by OProfile:/home/oprofile:/sbin/nologin
saslauth:x:498:76:"Saslauthd user":/var/empty/saslauth:/sbin/nologin
rpcuser:x:29:29:RPC Service User:/var/lib/nfs:/sbin/nologin
amandabackup:x:33:6:Amanda user:/var/lib/amanda:/bin/bash
dovenull:x:495:491:Dovecot's unauthorized user:/usr/libexec/dovecot:/sbin/nologin
user1:x:502:502::/home/user1:/bin/bash
user2:x:503:503::/home/user2:/bin/bash
user3:x:504:504::/home/user3:/bin/bash
user4:x:505:505::/home/user4:/bin/bash
user5:x:506:506::/home/user5:/bin/bash
user6:x:507:507::/home/user6:/bin/bash
user7:x:508:508::/home/user7:/bin/bash
user8:x:509:509::/home/user8:/bin/bash
user9:x:510:510::/home/user9:/bin/bash
user10:x:511:511::/home/user10:/bin/bash
user11:x:512:512::/home/user11:/bin/bash

grep user[1-5] /etc/passwd

user1:x:502:502::/home/user1:/bin/bash
user2:x:503:503::/home/user2:/bin/bash
user3:x:504:504::/home/user3:/bin/bash
user4:x:505:505::/home/user4:/bin/bash
user5:x:506:506::/home/user5:/bin/bash
user10:x:511:511::/home/user10:/bin/bash
user11:x:512:512::/home/user11:/bin/bash

La técnica de expresiones regulares.

Análisis de textos

cut -c1-10 /etc/passwd

root:x:0:0
bin:x:1:1:
daemon:x:2
adm:x:3:4:
lp:x:4:7:l
sync:x:5:0
shutdown:x
halt:x:7:0
mail:x:8:1
uucp:x:10:
operator:x
games:x:12
gopher:x:1
ftp:x:14:5
nobody:x:9
dbus:x:81:
rpc:x:32:3
usbmuxd:x:
abrt:x:173
pegasus:x:
cimsrvr:x:
hsqldb:x:9
oprofile:x
vcsa:x:69:
rtkit:x:49
avahi-auto
apache:x:4
saslauth:x
postfix:x:
rpcuser:x:
nfsnobody:
haldaemon:
gdm:x:42:4
tomcat:x:9
mysql:x:27
ntp:x:38:3
amandaback
memcached:
pulse:x:49
webalizer:
sshd:x:74:
postgres:x
dovecot:x:
dovenull:x
tcpdump:x:
Alumno:x:5
lcarrera:x
user1:x:50
user2:x:50
user3:x:50
user4:x:50
user5:x:50
user6:x:50
user7:x:50
user8:x:50
user9:x:51
user10:x:5
user11:x:5


Archivo de texto delimitado


Se va a comportar como una base de datos.

,; Windows
: Linux

[user2@luiscarrera ~]$ cut -d: -f1 /etc/passwd
root
bin
daemon
adm
lp
sync
shutdown
halt
mail
uucp
operator
games
gopher
ftp
nobody
dbus
rpc
usbmuxd
abrt
pegasus
cimsrvr
hsqldb
oprofile
vcsa
rtkit
avahi-autoipd
apache
saslauth
postfix
rpcuser
nfsnobody
haldaemon
gdm
tomcat
mysql
ntp
amandabackup
memcached
pulse
webalizer
sshd
postgres
dovecot
dovenull
tcpdump
Alumno
lcarrera
user1
user2
user3
user4
user5
user6
user7
user8
user9
user10
user11
[user2@luiscarrera ~]$ cut -d: -f1-3 /etc/passwd
root:x:0
bin:x:1
daemon:x:2
adm:x:3
lp:x:4
sync:x:5
shutdown:x:6
halt:x:7
mail:x:8
uucp:x:10
operator:x:11
games:x:12
gopher:x:13
ftp:x:14
nobody:x:99
dbus:x:81
rpc:x:32
usbmuxd:x:113
abrt:x:173
pegasus:x:66
cimsrvr:x:134
hsqldb:x:96
oprofile:x:16
vcsa:x:69
rtkit:x:499
avahi-autoipd:x:170
apache:x:48
saslauth:x:498
postfix:x:89
rpcuser:x:29
nfsnobody:x:65534
haldaemon:x:68
gdm:x:42
tomcat:x:91
mysql:x:27
ntp:x:38
amandabackup:x:33
memcached:x:497
pulse:x:496
webalizer:x:67
sshd:x:74
postgres:x:26
dovecot:x:97
dovenull:x:495
tcpdump:x:72
Alumno:x:500
lcarrera:x:501
user1:x:502
user2:x:503
user3:x:504
user4:x:505
user5:x:506
user6:x:507
user7:x:508
user8:x:509
user9:x:510
user10:x:511
user11:x:512
[user2@luiscarrera ~]$ cut -d: -f15,6 /etc/passwd
/root
/bin
/sbin
/var/adm
/var/spool/lpd
/sbin
/sbin
/sbin
/var/spool/mail
/var/spool/uucp
/root
/usr/games
/var/gopher
/var/ftp
/
/
/var/cache/rpcbind
/
/etc/abrt
/var/lib/Pegasus
/var/lib/Pegasus
/var/lib/hsqldb
/home/oprofile
/dev
/proc
/var/lib/avahi-autoipd
/var/www
/var/empty/saslauth
/var/spool/postfix
/var/lib/nfs
/var/lib/nfs
/
/var/lib/gdm
/usr/share/tomcat6
/var/lib/mysql
/etc/ntp
/var/lib/amanda
/var/run/memcached
/var/run/pulse
/var/www/usage
/var/empty/sshd
/var/lib/pgsql
/usr/libexec/dovecot
/usr/libexec/dovecot
/
/home/Alumno
/home/lcarrera
/home/user1
/home/user2
/home/user3
/home/user4
/home/user5
/home/user6
/home/user7
/home/user8
/home/user9
/home/user10
/home/user11

cut -d: -f16,7 /etc/passwd

/bin/bash
/sbin/nologin
/sbin/nologin
/sbin/nologin
/sbin/nologin
/bin/sync
/sbin/shutdown
/sbin/halt
/sbin/nologin
/sbin/nologin
/sbin/nologin
/sbin/nologin
/sbin/nologin
/sbin/nologin
/sbin/nologin
/sbin/nologin
/sbin/nologin
/sbin/nologin
/sbin/nologin
/sbin/nologin
/sbin/nologin
/sbin/nologin
/sbin/nologin
/sbin/nologin
/sbin/nologin
/sbin/nologin
/sbin/nologin
/sbin/nologin
/sbin/nologin
/sbin/nologin
/sbin/nologin
/sbin/nologin
/sbin/nologin
/sbin/nologin
/bin/bash
/sbin/nologin
/bin/bash
/sbin/nologin
/sbin/nologin
/sbin/nologin
/sbin/nologin
/bin/bash
/sbin/nologin
/sbin/nologin
/sbin/nologin
/bin/bash
/bin/bash
/bin/bash
/bin/bash
/bin/bash
/bin/bash
/bin/bash
/bin/bash
/bin/bash
/bin/bash
/bin/bash
/bin/bash
/bin/bash

[user2@luiscarrera ~]$ cut -d: -f3- /etc/passwd
0:0:root:/root:/bin/bash
1:1:bin:/bin:/sbin/nologin
2:2:daemon:/sbin:/sbin/nologin
3:4:adm:/var/adm:/sbin/nologin
4:7:lp:/var/spool/lpd:/sbin/nologin
5:0:sync:/sbin:/bin/sync
6:0:shutdown:/sbin:/sbin/shutdown
7:0:halt:/sbin:/sbin/halt
8:12:mail:/var/spool/mail:/sbin/nologin
10:14:uucp:/var/spool/uucp:/sbin/nologin
11:0:operator:/root:/sbin/nologin
12:100:games:/usr/games:/sbin/nologin
13:30:gopher:/var/gopher:/sbin/nologin
14:50:FTP User:/var/ftp:/sbin/nologin
99:99:Nobody:/:/sbin/nologin
81:81:System message bus:/:/sbin/nologin
32:32:Rpcbind Daemon:/var/cache/rpcbind:/sbin/nologin
113:113:usbmuxd user:/:/sbin/nologin
173:173::/etc/abrt:/sbin/nologin
66:65:tog-pegasus OpenPegasus WBEM/CIM services:/var/lib/Pegasus:/sbin/nologin
134:134:tog-pegasus OpenPegasus WBEM/CIM services:/var/lib/Pegasus:/sbin/nologin
96:96::/var/lib/hsqldb:/sbin/nologin
16:16:Special user account to be used by OProfile:/home/oprofile:/sbin/nologin
69:69:virtual console memory owner:/dev:/sbin/nologin
499:497:RealtimeKit:/proc:/sbin/nologin
170:170:Avahi IPv4LL Stack:/var/lib/avahi-autoipd:/sbin/nologin
48:48:Apache:/var/www:/sbin/nologin
498:76:"Saslauthd user":/var/empty/saslauth:/sbin/nologin
89:89::/var/spool/postfix:/sbin/nologin
29:29:RPC Service User:/var/lib/nfs:/sbin/nologin
65534:65534:Anonymous NFS User:/var/lib/nfs:/sbin/nologin
68:68:HAL daemon:/:/sbin/nologin
42:42::/var/lib/gdm:/sbin/nologin
91:91:Apache Tomcat:/usr/share/tomcat6:/sbin/nologin
27:27:MySQL Server:/var/lib/mysql:/bin/bash
38:38::/etc/ntp:/sbin/nologin
33:6:Amanda user:/var/lib/amanda:/bin/bash
497:495:Memcached daemon:/var/run/memcached:/sbin/nologin
496:494:PulseAudio System Daemon:/var/run/pulse:/sbin/nologin
67:67:Webalizer:/var/www/usage:/sbin/nologin
74:74:Privilege-separated SSH:/var/empty/sshd:/sbin/nologin
26:26:PostgreSQL Server:/var/lib/pgsql:/bin/bash
97:97:Dovecot IMAP server:/usr/libexec/dovecot:/sbin/nologin
495:491:Dovecot's unauthorized user:/usr/libexec/dovecot:/sbin/nologin
72:72::/:/sbin/nologin
500:500::/home/Alumno:/bin/bash
501:501::/home/lcarrera:/bin/bash
502:502::/home/user1:/bin/bash
503:503::/home/user2:/bin/bash
504:504::/home/user3:/bin/bash
505:505::/home/user4:/bin/bash
506:506::/home/user5:/bin/bash
507:507::/home/user6:/bin/bash
508:508::/home/user7:/bin/bash
509:509::/home/user8:/bin/bash
510:510::/home/user9:/bin/bash
511:511::/home/user10:/bin/bash
512:512::/home/user11:/bin/bash

Comandos básicos

id

Identificación del usuario

date

Fecha y hora del sistema

cal

Calendario del mes en curso

who

Muestra la lista de usuarios

w

Muestra la lista de usuarios

cat

Muestra el contenido del archivo

head

Muestra las diez primeras líneas del archivo

tail Muestras las diez últimas líneas del archivo

grep

Extrae líneas de un archivo que contengan una cadena específica o expresión regular

cut

Extrae columnas de un archivo o campos de un archivo de texto delimitado, el delimitador se indica con la opción d

wc (word count)

Cuenta el número de líneas, palabras, caracteres de un archivo de texto.

write

Envía un mensaje al terminal del usuario

talk

Inicia un chat entre dos usuarios activos.

Procedimiento

1. El usuario A inicia una conversación con el usuario B

talk B

2. El usuario B recibe una invitación para el chat, entonces debe responder con:

3. Se inicia la conversación
write user1

wall

 cat /etc/hosts
127.0.0.1   localhost localhost.localdomain localhost4 localhost4.localdomain4
::1         localhost localhost.localdomain localhost6 localhost6.localdomain6

[user2@luiscarrera ~]$ cal
    enero de 2018
lu ma mi ju vi sá do
 1  2  3  4  5  6  7
 8  9 10 11 12 13 14
15 16 17 18 19 20 21
22 23 24 25 26 27 28
29 30 31

[user2@luiscarrera ~]$ cal 2018
                               2018

        enero                 febrero                 marzo
lu ma mi ju vi sá do   lu ma mi ju vi sá do   lu ma mi ju vi sá do
 1  2  3  4  5  6  7             1  2  3  4             1  2  3  4
 8  9 10 11 12 13 14    5  6  7  8  9 10 11    5  6  7  8  9 10 11
15 16 17 18 19 20 21   12 13 14 15 16 17 18   12 13 14 15 16 17 18
22 23 24 25 26 27 28   19 20 21 22 23 24 25   19 20 21 22 23 24 25
29 30 31               26 27 28               26 27 28 29 30 31

        abril                  mayo                   junio
lu ma mi ju vi sá do   lu ma mi ju vi sá do   lu ma mi ju vi sá do
                   1       1  2  3  4  5  6                1  2  3
 2  3  4  5  6  7  8    7  8  9 10 11 12 13    4  5  6  7  8  9 10
 9 10 11 12 13 14 15   14 15 16 17 18 19 20   11 12 13 14 15 16 17
16 17 18 19 20 21 22   21 22 23 24 25 26 27   18 19 20 21 22 23 24
23 24 25 26 27 28 29   28 29 30 31            25 26 27 28 29 30
30
        julio                 agosto               septiembre
lu ma mi ju vi sá do   lu ma mi ju vi sá do   lu ma mi ju vi sá do
                   1          1  2  3  4  5                   1  2
 2  3  4  5  6  7  8    6  7  8  9 10 11 12    3  4  5  6  7  8  9
 9 10 11 12 13 14 15   13 14 15 16 17 18 19   10 11 12 13 14 15 16
16 17 18 19 20 21 22   20 21 22 23 24 25 26   17 18 19 20 21 22 23
23 24 25 26 27 28 29   27 28 29 30 31         24 25 26 27 28 29 30
30 31
       octubre               noviembre              diciembre
lu ma mi ju vi sá do   lu ma mi ju vi sá do   lu ma mi ju vi sá do
 1  2  3  4  5  6  7             1  2  3  4                   1  2
 8  9 10 11 12 13 14    5  6  7  8  9 10 11    3  4  5  6  7  8  9
15 16 17 18 19 20 21   12 13 14 15 16 17 18   10 11 12 13 14 15 16
22 23 24 25 26 27 28   19 20 21 22 23 24 25   17 18 19 20 21 22 23
29 30 31               26 27 28 29 30         24 25 26 27 28 29 30
                                              31

[user2@luiscarrera ~]$


 man cal
[user2@luiscarrera ~]$ cal 25 12 1
   diciembre de 1
lu ma mi ju vi sá do
          1  2  3  4
 5  6  7  8  9 10 11
12 13 14 15 16 17 18
19 20 21 22 23 24 25
26 27 28 29 30 31


cal 23 3 1964
    marzo de 1964
lu ma mi ju vi sá do
                   1
 2  3  4  5  6  7  8
 9 10 11 12 13 14 15
16 17 18 19 20 21 22
23 24 25 26 27 28 29
30 31

%%%%%%%%%%%%%%%%%%%%%%%%%%%%%%%%%%%%%%%%%%%%%%%%%


UNIX es privativo.

Solari es UNIX.

Linus Torvalds creó el Kernel de Linux.

Programa y libera el núcleo.

Linux es el núcleo del sistema operativo, una distribución es el núcleo de los programas.

Lo que se negocia es el soporte.
%%%%%%%%%%%%%%%%%%%%%%%%%%%%%%%%%%%%%%%%%%%%%%%%
Un terminal es una combinación "teclado + monitor".

Se necesita de un usuario, los terminales no tienen CPU ni memoria.

Los terminales deben de estar conectados a los sistemas de cómputo.


Vía SSH va encriptado la información entre el cliente y el servidor.
%%%%%%%%%%%%%%%%%%%%%%%%%%%%%%%%%%%%%%%%%%%%%%%%%%%

Cada puerto me identifica para poder brindar un servicio.

%%%%%%%%%%%%%%%%%%%%%%%%%%%%%%%%%%%%%%%%%%%%%%%%%%%
\end{document}

4 clases
Curso: Linux administración básica
Instructor: Luis Carrera / lcarrera@uni.edu.pe
Horario: SA 15:00 - 21:00
Duración 24 hrs / 4 clases 
Calificación: exámenes y ejercicios en clase
Nota mínima: 14